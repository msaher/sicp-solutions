\documentclass[12pt]{article}
\usepackage{amsmath}
\usepackage{amssymb}
\usepackage{hyperref} % for links

\begin{document}

\section*{Current interval arthmatic}

An identity $I$, such that for all intervals $x$, $x \times I = x$. Does
exit and has a value of $I = (1 , 1)$. \\ 

\emph{Proof}. For some positive-positive \( (x , y) \). The expression
\( (a , b) \times (x , y) \) is given by:

\begin{itemize}
	\item If \( (a , b) \) is positive-positive, then \( (a  , b) \times
		(x , y) = (a \cdot x , b \cdot y) \) 

	\item If \( (a , b) \) is negative-negative, then \( (a  , b) \times
		(x , y) = (a \cdot y , b \cdot x) \) 

	\item If \( (a , b) \) is negative-positive, then \( (a  , b) \times
		(x , y) = (a \cdot y , b \cdot y) \) 
\end{itemize}

Now, substituting \( (x , y) = (1 , 1) \). We see that all of the
expressions result in \( (a , b) \). Thus, \( (1 , 1) \) is the
multiplicative identity of intervals.  \hfill\(\Box\)
\\

However, there does not exist an inverse interval
$x^{-1}$, such that $x \times x^{-1} = (1,1)$.
\\

\emph{Proof}. Consider the multiplication of \( (lx , ux) \) and \( (ly
, uy). \) 

\begin{itemize}
	\item If \( x \)  is positive-positive, then

		\begin{itemize}
			\item If \( y \) is positive-positive \(
				\rightarrow (lx  \cdot ly , ux  \cdot
				uy) \).
			\item If \( y \) is negative-negative \(
				\rightarrow (ux
				\cdot ly, lx \cdot uy \).

			\item If \( y \) is negative-positive \(
				\rightarrow (ux
				\cdot ly, ux \cdot uy \).
		\end{itemize}

\end{itemize}

Examining the first case, if \( x \times y = (1 ,1) \implies (lx \cdot
ly , ux \cdot uy) = (1 , 1) \). This implies that \( y =
(\frac{1}{lx} , \frac{1}{ux})\). Since \( a \) must be less than \(
b \) in any given interval of the form \( (a , b) \). The
positive-positive inverse interval \( y \) does not exist. 

As for the rest of the cases, it can be shown that the inverse intervals
do not exist due to the fact that they can not produce a
positive-positive interval such as \( (1 , 1) \). Due to their sign.

Thus, for any given interval \( x \). An inverse does not necessarily 
exit.
\hfill \( \Box \)

Since the intervals do not produce a field. It's not possible to treat
them like real number. 

After some further research. I have encountered the following:
\href{https://en.wikipedia.org/wiki/Interval_arithmetic#Dependency_problem}{wiki page}

It seems that my approach was more mathematical, and I forgot what it
means to do asthmatic on intervals. In any case, Using the same symbol
more than one time, increases the error. See
\href{https://stackoverflow.com/questions/14130878/sicp-2-16-interval-arithmetic-scheme}{this}

\end{document}
