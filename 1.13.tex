\documentclass[13pt]{article}
\usepackage{amsmath,amsthm,amssymb}

\begin{document}

\section*{Proving \( f(n) = \frac{\phi ^{n} - \psi ^{n}}{\sqrt{5}
} \)}

\emph{Proof by induction.} The base case is for \( f(n) = 0 \) and \(
f(n+1) = 1 \).  It can be clearly seen that:

\begin{equation}
\frac{\phi ^{0} - \psi ^{0} }{\sqrt{5}} = 0 \text{ and
}\frac{\phi ^{1} - \psi ^{1}}{\sqrt{5}} = 1.
\end{equation}

Next, It must be shown that \( f(n+2) = f(n+1) + f(n) \) given
that \( f(n) = \frac{\phi ^{n} - \psi ^{n}}{\sqrt{5}} \) and \(
f(n+1) = \frac{\phi ^{n + 1} - \psi ^{n+1}}{\sqrt{5}} \).
By direct substitution : \( f(n+2) = \frac{\psi ^{n} - \psi
^{n}}{\sqrt{5}} + \frac{\phi ^{n+1} - \psi ^{n+1}  }{\sqrt{5}
} \). Which yields:

\begin{equation}
	f(n+2) = \frac{(\phi ^{n} + \phi ^{n+1} ) - (\psi ^{n} + \psi ^{n+1} )
	}{\sqrt{5}}
\end{equation}

This can be further reduced by taking \( n+2 \) as a common factor to
get \( f(n+2) = \frac{\phi ^{n+2} (\phi ^{-2} + \phi ^{-1}) - \psi ^{n+2} (\psi
^{-2}+ \psi ^{-1} )  }{\sqrt{5}}\). The result can be further simplified
as:

\begin{equation}
 f(n+2) = \frac{\phi ^{n+2} (\frac{\phi + 1}{\phi ^2}) - \psi ^{n+2}
(\frac{\psi + 1}{\psi ^2} )}{\sqrt{5}}.
\end{equation}

Recall that both \( \phi \) and \( \psi \) are the solution of the
equation \( x + 1 = x ^2 \). Therefore the following is true: 

\begin{itemize}
	\item \( \phi + 1 = \phi ^2 \) 
	\item \( \psi + 1 = \psi ^2 \)
\end{itemize}

Substitution with the above produces the result:

\begin{equation}
	f(n+2) = \frac{\phi ^{n+2} - \psi ^{n+2} }{\sqrt{5}} 
\end{equation}

As such, the induction process is complete. Therefore:

\begin{equation}
	f(n) = \frac{\phi ^{n} - \psi ^{n} }{\sqrt{5}} \label{proof1}
\end{equation}

Is true for all \( n \) in the domain of \( f \). As desired 
\hfill\(\Box\)

\section*{\( f(n) \) is the closest integer to \( \frac{\phi ^{n}
}{\sqrt{5}} \)}

\emph{Proof.} For \( f(n) \) to be the closest integer to \(
\frac{\phi ^{n}}{\sqrt{5}} \), the following inequality must be true:
\( \vert f(n) - \frac{\psi ^{n}}{\sqrt{5}} \vert \leq \frac{1}{2} \).

However, by equation (\ref{proof1}) we obtain \( \vert \frac{\phi ^{n}
- \psi ^{n} }{\sqrt{5}} - \frac{\phi ^{n}}{\sqrt{5}} \vert \leq
\frac{1}{2} \), which can be re-written as:

\begin{equation}
	\vert \psi \vert ^{n} \leq \frac{\sqrt{5}}{2} 
\end{equation}

It is known that \( \vert \psi \vert < 1 \). Therefore, \( \vert \psi \vert ^{n} < \vert
\psi ^{n-1} \vert \) for all \( n \in \mathbb{N} \), and since \( \vert
\psi \vert < \frac{\sqrt{5}}{2}\) it follows that \( \vert \psi
\vert ^{n}  < \frac{\sqrt{5}}{2} \), for all \( n \) in \(
\mathbb{N} \) 

\hfill$\Box$


\end{document}
